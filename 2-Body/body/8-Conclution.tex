\cleardoublepage
\chapter{总结与展望}
\section{总结研究}
物联网技术的迅猛发展带来了数据量的爆炸性增长,尤其是时序数据,这对存储系统提出了前所未有的挑战。
传统的集中式存储解决方案,尽管在管理和操作上相对简单,但它们存在明显的单点故障风险,一旦中心服务器发生故障,整个数据存储和处理流程可能会中断。
尽管分布式存储系统可以解决单点故障的问题,但是这些系统在数据安全性和不可篡改性方面仍然存在不足,特别是在需要高度透明度和安全性的应用场景中。
目前,区块链技术作为一种新兴的分布式存储技术,由于其去中心化、不可篡改和高度透明的特性,为物联网数据的安全存储和可信传输提供了新的解决方案。
然而,现有的区块链存储系统在处理物联网时序数据时仍然存在性能上的挑战,如存储延迟高、查询效率低、数据传输量大等。

针对这些挑战,本文提出了TimeChain,一种专为物联网时序数据设计的高效区块链存储技术。
TimeChain通过自适应聚合机制、基于共识的节点选择机制以及基于局部敏感哈希树的数据完整性验证机制,显著提升了存储效率和查询速度,同时确保了数据的安全性和完整性。TimeChain的设计不仅解决了传统存储系统的局限性,也为物联网数据的高效可信存储提供了新的解决方案。

本文的主要研究工作可以概述如下:

\begin{itemize}
    \item 在\textbf{数据聚合}阶段,本研究提出了一个针对链下存储环境的自适应聚合策略。
    为了降低在执行范围查询时需要访问的存储节点数目,本文构建了一个自适应无向加权图,将数据批处理问题转化为图的分割问题。
    鉴于物联网数据查询的不规则性,传统的聚类方法如K-means和GMM并不适用,因其或需预设图的形状和数量,或计算复杂度过高。
    因此,本文选择了谱聚类算法来处理子图的划分,以减少在聚合查询过程中访问节点的次数。
    \item 在\textbf{存储节点选择}阶段,本文提出了一个依托共识机制的存储节点选择方案。
    在此过程中,本文重点关注了选择的准确性和安全性。
    准确性涉及选择的节点应具有较短的传输延迟和较高的存储服务质量,而安全性则意味着选择过程应能抵御单点故障或恶意节点的干扰。TimeChain在此方面综合考虑了节点的信誉度、可用存储空间和传输距离。
    同时,为了在去中心化环境中高效地进行节点选择,本文将共识机制与节点选择过程相结合,以降低传播延迟。
    \item 在\textbf{数据验证}阶段,本文设计了一种基于局部敏感哈希树的数据完整性验证机制。
    传统的默克尔树在构建过程中会产生与数据量相等的哈希值,导致传输成本过高。
    针对这一问题,本文的LSH树机制利用物联网数据中时间序列点的相似性,仅传输非冗余部分,从而显著降低了验证所需的数据量。
\end{itemize}

基于开源组件实现的TimeChain,在性能评估中显示出了卓越的性能。与现有的区块链存储系统相比,TimeChain平均减少了64.6\%的查询延迟和35.3\%的存储延迟,证明了其在提升物联网时序数据处理效率方面的明显优势。
此外,TimeChain在支持的最大存储设备数量方面也显示出了优秀的扩展性,与SEBDB和FileDES相比,分别提升了1.63倍和3.55倍。
这表明TimeChain能够适应日益增长的物联网设备数量,满足未来物联网应用的发展需求。
通过消融实验,本文进一步证实了TimeChain中自适应聚合机制、基于共识的节点选择机制和基于LSH树的验证机制对于系统性能提升的关键作用。

\section{未来工作}
在未来的工作中,本研究计划针对TimeChain系统进行以下几方面的研究和改进,以进一步提升系统性能并适应不同的应用场景。

目前,TimeChain的测试和评估工作仅在模拟环境中进行,这限制了对其在现实世界条件下性能的全面理解。
未来的工作将从模拟环境转向实际应用场景,以全面评估TimeChain的性能和可靠性。
这意味着TimeChain将被部署在真实环境中,从而验证其在现实世界条件下的实际表现,确保其能够满足各种应用场景的需求。
这包括在不同网络条件下对TimeChain算法的响应时间和吞吐量进行详细评估,以确保其在实际应用中的效率和稳定性。
通过这种方式,可以更准确地理解TimeChain在面对真实网络波动和数据流量时的行为,从而为进一步的优化和改进提供实际依据。

其次,针对PBFT算法在高吞吐量场景下可能遇到的性能瓶颈,本文计划研究并引入更轻量级或异步的共识算法,例如Raft和Tendermint。
这些算法被选中是因为它们在保持安全性的同时,能够有效减少通信开销,提升系统的响应速度和处理能力。
这对于资源受限的物联网设备尤为重要,因为它们往往对能耗和计算能力有严格的限制。
此外,未来的研究将探索分片技术和侧链技术等其他提升区块链性能的方法,以增强TimeChain的可扩展性,并深入分析这些技术可能引入的安全风险,制定相应的对策。

最后,由于在TimeChain中引入了新共识算法,TimeChain的安全性和稳定性将成为本研究关注的焦点。
在未来的工作中,本研究将对所提出的共识协议和基于LSH树的数据完整性验证机制进行严格的实验验证。
这包括在模拟存在欺诈节点的物联网环境中进行实验,以评估TimeChain在面对恶意攻击时的表现,以及其保护数据完整性和用户隐私的能力。通过这些实验,本研究旨在确保TimeChain在提供高效存储解决方案的同时,也能保障数据的安全性和隐私性,满足物联网领域对高安全标准的严格要求。

综上所述,未来的工作将围绕TimeChain的实际部署、性能优化和安全性验证展开,以确保该系统能够在多样化的应用场景中提供可靠、高效和安全的数据存储服务。
