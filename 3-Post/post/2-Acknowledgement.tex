\cleardoublepage
\chapternonum{致谢}
这段求学时光宛若一场跌宕起伏的旅程,充满了各种滋味。
有过攻克难题后的欣喜若狂,有过迷茫探索时的黯然神伤,也有过压力重重下的咬牙坚持,更有收获成果时的心满意足。
如今这段旅程即将画上句号,那些在途中给予我援手的人,我必须诚挚地道一声感谢。

感谢EmNets实验室的导师们的悉心指导和帮助。
董玮老师高屋建瓴的思维和严谨的治学态度,为我的学术探索点亮了high-level的灯塔,避免我陷入细节主义的泥淖。
高艺老师脚踏实地的指导和细致的科研态度,为我夯实了科研之路上的每一步,使我在探索未知的道路上稳步前行。
吕嘉美老师,或者应该喊吕博,则是我亦师亦友的好伙伴。
我时常怀念起我们椅子紧靠头脑风暴、吃饭带上草稿本和笔一起讨论技术路线的日子。
在我因为赶deadline草木皆兵、因为rebuttal压力山大的时候,是她一直陪在我身边,教我怎么左右手互博,怎么和审稿人斗智斗勇。

感谢EmNets的各位小伙伴:
李烨明、孙桐、肖凯杰、黄家名、郭呈、金乐伟、李仁杰、魏青欣、苏静怡
彭颖琦、龚凯杰、张洋、林海龙、尤新云、李福、姜博文、杨睿祈、张海浪、孔博宇、徐伟峰、赵子涵、陈鹏郅、孟特石、
王浩、陈永麒、王子平、潘福长、吴勋宇、张魁、卓一帆、李乐翔、胡凯、张佳楠、刘洋、宋宇明。
和大家一起讨论一起玩乐的日子永远会在我灰暗的科研苦旅中熠熠闪光。
感谢酒鬼群、棋牌室小队、羽毛球小队的各位,在我低落的时候一起伸出手,把我从阴暗的角落里拉了出来。
我还记得那个夏天的晚上,他们大老远哼哧哼哧骑车来喊我吃夜宵的样子,那是我觉得最丰盛的夜宵了。
感谢已经毕业的张文照、李博睿、周寒、吴昊、丁智、李皓宇、周宏、段瑶光、曹丁越、徐诚阳、李经纬、聂志康、刘荣盛,
在我刚入学懵懂且迷茫的时候,他们扶着我的肩膀,告诉我应该如何触摸科研;
在我因为秋招而手足无措的时候,他们摸着我的脑袋,告诉我应该如何直通offer。
感谢张嘉懿、排球院队、435的各位,一同经历的青春岁月,是我人生中最美好的回忆。

感谢我的母亲,她是我生命中最重要的人,她的包容和支持是我前进的动力。
在我因为科研压力而焦虑的时候,她总是在家里准备好一桌饭菜,忍受我无声的苦闷,
绞尽脑汁寻找轻松的话题,四处探寻能让心灵放松的去处,只为给我带来片刻的安宁与愉悦。
感谢我的父亲,尽管总是聚少离多,他总是用他那瘦削的肩膀,为我撑起一片坚实的天空。
感谢我的亲人们,彼时我总把他们投来的目光当成必须成才的殷切期望,如今我明白,那是他们对我的无限信任和支持。

感谢我的对象王耀仑,他是我决定要牵手一起走下去的人。
在我患得患失、焦虑不安的时候,他总是给我安慰,告诉我不要害怕。
在我反复打磨论文的无趣时光里,他总是不厌其烦地给出语法和行文逻辑上细致入微的建议。
我知道他总会在那里等着我。

我要感谢那天晚上吹过曹西204的风,让我拿起铅笔橡皮推导EdgeRuler算法的手不再颤抖。

我要感谢那天清晨实验室窗外的朝霞,让我TimeChain赶deadline和rebuttal的心不再动摇。

感谢这个努力的自己。

\vspace{\baselineskip}

\noindent\hfill 滕依筱

\noindent\hfill 二零二五年三月于求是园 \\
