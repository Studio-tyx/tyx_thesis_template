\cleardoublepage
\chapternonum{摘要}

与传统集中式存储相比,基于区块链的去中心化存储系统提供了更高的安全性、透明度和更低的成本,使其成为点对点协作的理想选择。
然而,较低的事务处理速度和较高的计算要求限制了它在物联网时序数据等高密度数据场景中的应用。

为解决现有区块链存储方案在物联网时序数据场景下的低效问题,本文基于链下存储方案提出了一种面向物联网时序数据的基本存储架构。
该系统对离散时序数据进行批处理,并仅将每个批次的哈希值存储在链上,而完整的数据则保留在链下。
这种方法显著降低了区块链上的存储开销,将存储延迟减少到原先的三十七分之一。

然而,这种基本架构在查询性能方面存在不足。
为此,本文进一步提出了TimeChain,一个基于区块链的物联网时序数据高效可信存储系统。
为了减少范围查询中的传输延迟,TimeChain引入了自适应打包机制。
通过将数据和历史协同查询特征分别表示为图的顶点和边权重,本文将批处理问题转化为图划分问题,显著提高了查询效率。
为了减少节点选择过程中的时延,TimeChain将节点选择和共识过程结合,提出了基于共识协议的节点选择机制。
为了最小化数据完整性验证中的传输量,TimeChain采用了一种基于局部敏感哈希树的数据完整性校验机制。
该机制通过仅传输非冗余部分来最大限度地减少完整性检查所需的数据量。

本文基于开源组件实现了TimeChain,并对其性能进行了评估。
实验结果显示,与现有的基于区块链的存储系统相比,TimeChain平均减少了64.6\%的查询延迟和35.3\%的存储延迟。
这表明TimeChain在提高物联网时序数据处理效率方面具有显著优势。

\textbf{关键词:} 物联网、区块链、数据存储、时序数据